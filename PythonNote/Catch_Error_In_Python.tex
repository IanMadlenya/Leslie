%!TEX encoding = UTF-8 Unicode
% $Id$
% Catch_Error_In_Python.tex

\documentclass[11pt,a4paper]{article}
\usepackage{CJK}

%如果pdf制作中文书签有乱码,则用下面命令
\usepackage[pdftex,   %pdflatex,pdftex,dvipdfm
            pdfstartview=FitH,
            CJKbookmarks=true,
            bookmarksnumbered=true,
            bookmarksopen=true,
            pdfborder=001,   %注释掉此行,则交叉引用为彩色边框
            colorlinks,
            linkcolor=blue,
            anchorcolor=green,
            citecolor=green]{hyperref}
%\usepackage{ccaption} %对图的编号用中文显示
%\usepackage{makeidx}  %自动产生索引
%\usepackage{hyperref}
%\usepackage[colorlinks,linkcolor=red,anchorcolor=blue,citecolor=green]{hyperref}

\begin{document}
\begin{CJK*}{UTF8}{nsung}

\title{异常处理:Catch Error In Python}
\date{Sat Dec 22,2012}
\author{Leslie Zhu}
\maketitle

\begin{center}
Copyright \copyright \ 2012 Leslie Zhu (pythonisland@gmail.com)
\end{center}

\newpage

这是我关于《Python核心编程》异常处理章节的摘录笔记,写这篇笔记目的有二:一是实践\LaTeX语法,二是学习下''with语句上下文管理''的使用。\par

\section{什麽是异常}

\subsection{错误}
~~~~错误分为语法错误和逻辑错误等,语法错误由编译器或解释器检查,逻辑错误则需要debug;当Python解释器检查到错误时,则出现了异常,导致程序无法继续运行。\par
\subsection{异常}

\section{Python中的异常}

\subsection{NameError:尝试访问一个未申明的变量}
\begin{verbatim}
>>> foo
Traceback (innermost last):
 File "<stdin>", line 1,in ?
NameError: name 'foo' is not defined
\end{verbatim}

\subsection{ZeroDivisionError: 除数为零}
\begin{verbatim}
>>> 1/0
Traceback (innermost last):
  File "<stdin>", line 1, in ?
ZeroDivisionError: integer division or module by zero
\end{verbatim}

\subsection{SyntaxError: Python解释器语法错误}
\begin{verbatim}
>>> for
   File "<string>", line 1
    for
      ^
SyntaxError: invalid syntax
\end{verbatim}

\subsection{IndexError: 请求的索引超出序列范围}
\begin{verbatim}
>>> aList = list()
>>> aList[0]
Traceback (innermost last):
  File "<stdin>", line 1, in ?
IndexError: list index out of range
\end{verbatim}

\subsection{KeyError: 请求一个不存在的字典关键字}
\begin{verbatim}
>>> aDict = {'host': 'earth', 'port': 80}
>>> print aDict['server']
Traceback (innermost last):
   File "<stdin>", line 1, in ?
KeyError: server
\end{verbatim}

\subsection{IOError: 输入/输出错误}
\begin{verbatim}
>>> f = open("blah")
Traceback (innermost last)
   File "<stdin>", line 1, in ?
IOError: [Errno 2] No such file or directory: 'blah'
\end{verbatim}

\subsection{AttributeError: 尝试访问未知的对象属性}
\begin{verbatim}
>>> class myClass(object):
...    pass
...
>>> myInst = myClass()
>>> myInst.bar = 'spam'
>>> myInst.bar
'spam'
>>> myInst.foo
Traceback (innermost last):
   File "<stdin>", line 1, in ?
AttributeError: foo
\end{verbatim}

\subsection{标准异常}

~~~~表1~列出所有的Python标准异常,所有异常都是内建(built-in),不包括用户自定义异常。


\begin{table}[!hbp]
\caption{Python标准异常}
\begin{tabular}{|c|l|}
\hline
Name                              & 描述\\
\hline
BaseException                     & 所有异常的基类\\
\hline
SystemExit                        & python解释器请求退出\\
\hline
KeyboardInterrupt                 & 用户终端执行(Ctrl+C)\\
\hline
Exception                         & 常规错误的基类\\
\hline
StopIteration                     & 迭代器没有更多的值\\
\hline
GeneratorExit                     & 生成器发生异常\\
\hline
StandardError                     & 所有的内建标准异常的基类\\
\hline
ArithmeticError                   & 所有数值计算错误的基类\\
\hline
FloatingPointError                & 浮点计算错误\\
\hline
OverflowError                     & 堆栈溢出错误\\
\hline
ZeroDivisionError                 & 除(或取模)零\\
\hline
AssertionError                    & 断言语句失败\\
\hline
EOFError                          & 没有内建输入,到达EOF标记\\
\hline
AttributeError                    & 访问对象没有的属性错误\\
\hline
EnvironmentError                  & 操作系统环境错误的基类\\
\hline
IOError                           & 输入/输出错误\\
\hline
OSError                           & 操作系统错误\\
\hline
WindowsError                      & Windows系统调用失败\\
\hline
ImportError                       & 导入模块对象失败\\
\hline
LookupError                       & 无效的数据查询的基类\\
\hline
IndexError                        & 访问索引越界错误\\
\hline
KeyError                          & 访问Key失败\\
\hline
MemoryError                       & 内存溢出错误\\
\hline
NameError                         & 未声明/初始化对象\\
\hline
UnboundLocalError                 & 访问未初始化的本地变量\\
\hline
ReferenceError                    & 弱引用试图访问已经垃圾回收的对象\\
\hline
RuntimeError                      & 一般的运行时错误\\
\hline
NotImplementedError               & 尚未实现的方法\\
\hline
SyntaxError                       & Python语法错误\\
\hline
IndentationError                  & 缩进错误\\
\hline
TabError                          & Tab和空格混用\\
\hline
SystemError                       & 一般的解释器系统错误\\
\hline
TypeError                         & 对类型无效的操作\\
\hline
ValueError                        & 传入无效参数\\
\hline
UnicodeError                      & Unicode相关错误\\
\hline
UnicodeDecodeError                & Unicode解码错误\\
\hline
UnicodeEncodeError                & Unicode编码错误\\
\hline
UnicodeTranslateError             & Unicode转换时错误\\
\hline
Warning                           & 警告的基类\\
\hline
DeprecationWarning                & 关于被弃用的特征的警告\\
\hline
FutureWarning                     & 关于构造将来语义会有改变的警告\\
\hline
OverflowWarning                   & 关于自动提升为长类型的警告\\
\hline
PendingDeprecationWarning         & 关于特性将会被遗弃的警告\\
\hline
RuntimeWarning                    & 可疑的运行时警告\\
\hline
SyntaxWarning                     & 可疑的语法警告\\
\hline
UserWarning                       & 用户代码生成的警告\\
\hline
\end{tabular}

\end{table}

\newpage

\subsection{字符串作为异常}

~~~~Python 1.5之前,标准的异常是基于字符串实现的,导致异常直接不能有相互关系,之后采用类实现标准异常。

\section{触发异常}

Python提供一种机制让程序员主动触发异常,即raise语句。

\subsection{语法和习惯用法}

raise一般用法:
\begin{verbatim}
raise [SomeException [, args [, traceback]]]
\end{verbatim}
第一个参数,SomeException是触发异常的名字;如果有此参数,必须是一个字符串、类或实例。\\
第二个参数可选,args传给异常。如果args是一个单独对象,则生成只有一个元素的元组;如果args是元组,则直接传给异常进行处理。\\
第三个参数可选,traceback是当异常触发时新生成的一个用户异常-正常化(exception-normally)的跟踪记录对象。\\
最常见的用法是SomeException是一个类。

\subsection{特殊用法}

如果第一个参数不是类,而是一个实例,并且不是异常类及其子类的实例时,会发生什麽情况? 
\begin{center}
\begin{table}[!hbp]
\caption{raise语法}
\begin{tabular}{|l|l|}
\hline
raise语法                         & 描述\\
\hline
raise exclass                     & 触发一个异常,从exclass生成一个实例(不含任何参数)\\
\hline
raise exclass()                   & 通过函数调用操作符作用于类名生成一个新的exclass实例,没有参数\\
\hline
raise exclass, args               & 提供参数args,可以是一个参数或元组\\
\hline
raise exclass(args)               & 同上\\
\hline
raise exclass, args, tb           & 同上,提供跟踪记录对象tb\\
\hline
                                  & 通过实例触发异常(通常是exclass实例);如果实例是exclass的\\
raise exclass, instance           & 子类实例,那么这个新异常的类型会是子类的类型(而不是exclass);\\
                                  & 如果实例既不是exclass的实例也不是其子类\\
                                  & 的实例,则复制此实例为异常参数去生成一个新的exclass实例\\
\hline
raise string                      & (过时的)触发字符串异常\\
\hline
raise string, args                & 同上\\
\hline
raise string, args, tb            & 同上\\
\hline
raise                             & 重新触发上一个异常;如果之前没有触发异常,触发TypeError异常\\
\hline
\end{tabular}
\end{table}
\end{center}
\newpage

\section{断言}

类似C语言的assert宏,即测试一个表达式,返回False,则触发AssertionError异常。

\subsection{断言语句}

assert 语法:
\begin{verbatim}
assert expression[, argument]
\end{verbatim}
比如:
\begin{verbatim}
assert 1 == 1
assert 2 + 2 == 2 * 2
assert len(['my list',12]) < 10
assert range(3) == [0, 1, 2]
\end{verbatim}
Python的一个assert实现:
\begin{verbatim}
def assert(expr, args=None):
    if __debug__ and not expr:
        raise AssertionError, args
\end{verbatim}
内建变量\_\_debug\_\_在通常情况是True,如果开启优化(\-O)后则为False。


\section{创建异常}

尽管标准异常集包括广泛的异常类型,但有时用户需要自定义异常以明确异常的具体问题来源,这里举例创建NetworkError异常。\par

\begin{verbatim}
#!/usr/bin/env python
#-*- coding: utf-8 -*-

import os, socket, errno, types, tempfile

class NetworkError(IOError):
    pass

def updArgs(args, newarg=None):
    if isinstance(args, IOError):
        myargs = []
        myargs.extend([arg for arg in args])
    else:
        myargs = list(args)
    if newarg:
        myargs.append(newarg)
    
    return tuple(myargs)

def myconnect(sock, host, port):
    try:
        sock.connect((host,port))
    except socket.error, args:
        myargs = updArgs(args)
        if len(myargs) == 1:
            myargs = (errno.ENXIO, myargs[0])
    
        raise NetworkError, updArgs(myargs, host + ': ' + str(port))

def testnet():
    s = socket.socket(socket.AF_INET, socket.SOCK_STREAM)
    for eachHost in ('deli', 'www'):
        try:
            myconnect(s, 'deli', 8080)
        except NetworkError, args:
            print "%s: %s" % (args.__class__.__name__, args)

if __name__ == '__main__':
    testnet()

出错信息:

NetworkError: [Errno 113] No route to host: 'deli: 8080'
NetworkError: [Errno 113] No route to host: 'deli: 8080'
\end{verbatim}

\section{为什麽要用异常}

~~~~专业的程序员面对异常,希望能从异常报错信息中找到错误的来源,而对于普通用户来说,程序报错意味着怀疑和不安全,从而失去生意和信任。对于普通用户来说,一般的错误是因为输入数据无效导致的;程序应该能够处理着各种用户级错误,当错误发生时能继续运行并成功完成,向用户返回一个有效的说明信息,而不是返回一堆专业人员查看的代码。\par

对于Python语言,如果没有异常处理,编码人员不得不时刻为各种错误增加繁杂冗余的代码;异常不仅简化代码,而且简化整个错误管理体系。

\section{异常和sys模块}
~~~~还有一种获取异常信息的途径,通过sys模块的exc\_info()函数,此函数提供一个三元组信息:
\begin{verbatim}
>>> try:
...     float('abc123')
... except:
...     import sys
...     exc_tuple = sys.exc_info()
... 
>>> for eachItem in exc_tuple:
...     print eachItem
...     
... 
<type 'exceptions.ValueError'>
could not convert string to float: abc123
<traceback object at 0xa2812d4>
>>>
\end{verbatim}

\section{检测和处理异常}

~~~~异常通过try语句检查,任何在try语句块里的代码都会检查有无异常发生。\\

~~~~try语句有两种主要形式:try-except和try-finally。这两个语句是互斥的,只能一次使用一种。一个try语句可以对应一个或多个except子句,但只能对应一个finally子句,或是一个try-except-finally符合语句。

\subsection{try-except语句}
\begin{verbatim}
try:
    try_suite      #监控这里的代码
except Exception[, reason]:
    except_suite   #异常处理代码
\end{verbatim}

\subsection{多个except的try语句}
\begin{verbatim}
try:
    try_suite        
except Exception1[, reason]:
    except_suite_1   
except Exception2[, reason]:
    except_suite_2
\end{verbatim}

\subsection{处理多个异常的except语句}
\begin{verbatim}
except (Exception1, Exception2) [, reason]:
    suite_for_Exception1_and_Exception2
\end{verbatim}

\subsection{捕获所有异常}
\begin{verbatim}
try:
    :
except Exception, e:
    # error occurred, log 'e', etc.

不推荐采用这个不"Pythonic"的用法:
try:
    :
except:
    # error occurred, etc.
\end{verbatim}
~~~~关于捕获所有异常,对于SystemExit和KerboardInterrupt不是由于错误条件引起的。SystemExit是因为当前Python应用程序需要退出,而KeyboardInterrupt是用户按下Ctrl+C关闭Python。在真正需要的时候,这些异常会当作异常处理捕获,一个典型的迂回工作法代码框架可能会这样:
\begin{verbatim}
try:
    :
except (KeyboardInterrupt, SystemExit):
    # user wants to quit
    raise                      # reraise back to caller
except Exception:
    #handle real errors
\end{verbatim}

从Python2.5开始,“所有异常之母” BaseException是所有异常的基类,异常的结构是:
\begin{verbatim}
-BaseException
 |-KeyboardInterrupt
 |-SystemExit
 |-Exception
   |-(all other current build-in exceptions)
\end{verbatim}
~~~~所以当有Exception处理器后,不需要为SystemExit和KeyboardInterrupt创建额外的处理器了:
\begin{verbatim}
try:
    :
except Exception, e:
    # handler real errors
\end{verbatim}

如果确实需要捕获所有的异常,采用新的BaseException:
\begin{verbatim}
try:
    :
except BaseException, e:
    # handler real errors
\end{verbatim}

\subsection{异常参数}

~~~~当在自己代码引发内建异常时,尽量遵循规范,用和已有Python代码一致错误信息作为异常的参数元组的一部分。简单说,如果因为一个ValueError,最好提供和解释器引发ValueError时一致的参数信息,以此类推。这样可以保证代码一致性,同时也能避免其它应用程序在使用此模块时发生错误。\par


\subsection{else子句}

~~~~在try范围中没有异常被检查到时,执行else子句。在else返回的任何代码运行前,try范围中的所有代码必须完全成功(即结束前,没有引发异常)。

\subsection{finally子句}
~~~~finally子句是无论异常是否发生、是否捕获都会执行的一段代码:
\begin{verbatim}
try:
    A
except MyException:
    B
else:
    C
finally:
    D

等价旧版本的:
try:
    try:
        A
    except MyException:
        B
    else:
        C
finally:
    D
\end{verbatim}

\subsection{try-finally子句}
~~~~这个子句不是用来捕获异常的,而是用来维持一致的行为,即不管异常是否发生:
\begin{verbatim}
try:
    try_suite
finally:
    finally_suite
\end{verbatim}

\subsection{try-except-else-finally:无一漏网}
\begin{verbatim}
try:
    try_suite
except Exception1:
    suite_for_Exception1
except (Exception2, Exception3):
    suite_for_Exception2_and_3
except Exception3, e:
    suite_for_Exception3_plus_args
except (Excetpion4,Exception5),e:
    suite_for_Exception4_Exception5_plus_args
except:
    suite_for_all_other_exceptions
else:
    no_exceptions_detected_suite
finally:
    always_excute_suite
\end{verbatim}

\section{上下文管理}
\subsection{with语句}

~~~~类似try-except-finally和try-finally语句一样,with语句也是用来简化代码的。try-except和try-finally的一种特定的配合用法是用来保证共享资源的唯一分配,并在任务结束后释放。比如文件、线程资源、简单同步、数据库连接等,with语句的目标就是应用在这种场景。\\

但with语句的目的在于从流程图中把try、except和finally关键字和资源分配释放相关代码统统去掉,而不是仅仅简化代码。\\
with语法:
\begin{verbatim}
with context_expr [as var]:
    with_suite
\end{verbatim}
~~~~with语句仅能工作于支持上下文管理协议(context management protocol)的对象,即只有内建了“上下文管理”的对象可以和with一起工作。\\
支持该协议的对象列表:
\begin{verbatim}
* file
* decimal.Context
* thread.LockType
* threading.Lock
* threading.RLock
* threading.Semaphore
* threading.BounderSemaphore
\end{verbatim}
比如:
\begin{verbatim}
with open('/etc/passwd','r') as f:
    for eachLine in f:
        #...do stuff with eachLine or f...
\end{verbatim}
with语句会完成准备工作,不论是代码开始、中间或结束出现异常,都会自动执行清理代码。

\subsection{上下文管理协议}

~~~~除非打算自定义可以和with一起工作的类,大多数Python程序员仅仅需要知道with语句如何使用。
\subsubsection{上下文表达式(context\_expr),上下文管理器}

~~~~当with语句执行时,会执行上下文符号(即with和as中间的内容)来获得一个上下文管理器。上下文管理器的职责是提供一个上下文对象,通过调用\_\_context\_\_()方法实现。
\subsubsection{上下文对象,with语句块}

~~~~一旦获得上下文对象,会调用它的\_\_enter()\_\_方法,完成with语句块执行前的所有准备工作。当with语句块结束(无论怎么结束),调用上下文对象的\_\_exit()\_\_方法。此方法有三个参数,如果with语句块正常结束,三个参数都是None;如果发生异常,则三个参数的值分别等于调用sys.exc\_info()函数返回的值:类型、值、跟踪记录。




\section{再次讨论with语句}

~~~~根据\url{www.python.org/dev/peps/pep-0343}{}描述:
\begin{verbatim}
This PEP adds a new statement "with" to the Python language to make
it possible to factor out standard uses of try/finally statements.

In this PEP, context managers provide __enter__() and __exit__()
methods that are invoked on entry to and exit from the body of the
with statement.
\end{verbatim}
with语句是让用户从try/finally语句中解放出来,上下文管理器提供入口、出口函数\_\_enter\_\_(),\_\_exit\_\_()。\\
比如下面的with语法:
\begin{verbatim}
    with VAR = EXPR:
        BLOCK
等同于
    VAR = EXPR
    VAR.__enter__()
    try:
        BLOCK
    finally:
        VAR.__exit__()
\end{verbatim}
比如定义一个上下文管理器:

\end{CJK*}
\end{document}
